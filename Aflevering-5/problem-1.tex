\section*{Problem 1}

The task is to count the number of letters given as arguments to the program.

To do this I assume the input to be ASCII and thus all latin uppercase letters
are number 65 up to and including 90 and all lowercase letters are from 97 up
to and including 122. Thus I only count characters for which the following is
true:
$$ \text{char} \in [65;90] \text{ or } [97;122] $$
or said in another way
$$ \text{char} \in [A;Z] \text{ or } [a;z] $$

The \code{main}-function receives an argument named \code{argv} which is a
list of arguments to the program. Thus it is a list of lists of chars. I
iterate over all items of the outermost list. For each list of chars I check
every character to see if it is a letter. When a \code{\textbackslash0} is encountered
I've reached the end of the list/string and must then continue on the next
argument.

The program will ignore any letter which isn't from a-z or A-Z. Thus local
symbols (æ, ø, å, $\phi$, ... ) and accented symbols (è, é, ... ) are ignored.

Some example runs

\begin{verbatim}
> ./letters.o abzAZ
5

> ./letters.o abz AZ
5

> ./letters.o a-b
2

> ./letters.o æøå
0
\end{verbatim}
