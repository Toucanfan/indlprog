\section*{Problem 2}
The stack a chunk of memory which primarily stores local variables in programs.
The stack is a FILO (first in, last out) data structure, due to the nested way in which programs typically assign temporary variables; functions with local variables calling more functions which themselves declare local variables. 
In order to replicate the way the stack operates, the following functions must be available:

\begin{itemize}
\item \code{push}, which puts data onto the stack.
\item \code{top}, which reads the last entered data in the stack.
\item \code{pop}, which reads \textit{and} removes the top of the stack.
\item \code{empty}, which simply checks if the stack is currently empty.
\end{itemize}

In this implementation, the memory portion of the stack is merely an array that dynamically expands, in our case by a factor of two, every time there is insufficient space, as indicated by the test run, which fills the stack with 9 integers and then repeatedly \code{pop}'s it afterwards, until \code{empty} indicates there are no more elements left in the stack.

\begin{verbatim}
    Capacity:  1, Size:  0, contents: []
    Capacity:  1, Size:  1, contents: [0]
    Capacity:  2, Size:  2, contents: [0,1]
    Capacity:  4, Size:  3, contents: [0,1,2]
    Capacity:  4, Size:  4, contents: [0,1,2,3]
    Capacity:  8, Size:  5, contents: [0,1,2,3,4]
    Capacity:  8, Size:  6, contents: [0,1,2,3,4,5]
    Capacity:  8, Size:  7, contents: [0,1,2,3,4,5,6]
    Capacity:  8, Size:  8, contents: [0,1,2,3,4,5,6,7]
    Capacity: 16, Size:  9, contents: [0,1,2,3,4,5,6,7,8]
    Capacity: 16, Size:  8, contents: [0,1,2,3,4,5,6,7]
    Capacity: 16, Size:  7, contents: [0,1,2,3,4,5,6]
    Capacity: 16, Size:  6, contents: [0,1,2,3,4,5]
    Capacity: 16, Size:  5, contents: [0,1,2,3,4]
    Capacity: 16, Size:  4, contents: [0,1,2,3]
    Capacity: 16, Size:  3, contents: [0,1,2]
    Capacity: 16, Size:  2, contents: [0,1]
    Capacity: 16, Size:  1, contents: [0]
    Capacity: 16, Size:  0, contents: []
\end{verbatim}

