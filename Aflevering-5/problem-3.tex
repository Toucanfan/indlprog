\section*{Problem 3}
\subsection*{Specification}
The task was to write an interactive console application, \code{cudb},
which can be used for managing university students. The user must be able
to add new students (name, start year, start semester) as well as
list all students. The student data must be saved in a special manner,
which is detailed in the exercise document.

\subsection*{Design and implementation}
The source code consist of 6 files, whose functionality now will be described.

\subsubsection*{\code{Makefile}}
This file describes the dependencies between the application
source files, such that one can invoke the GNU Make program to build the application
instead of manually compiling and linking each file in the correct order.

\subsubsection*{\code{main.c}}
This file contains the entry point for the whole application: the \code{main()}-
function. This function contains the application's event loop, which will call various
subroutines in the same file depending on what the user wants to do. These subroutines will
in turn make use of the interfaces provided by the other files to communicate with the user
and manage students.

\subsubsection*{\code{cli.h}}
This file defines the interface (function prototypes) through 
which other parts of the program (\code{main.c}) can communicate 
with the user. The functions can prompt the user for various types
of input and do input validation as well as print information.

\subsubsection*{\code{cli.c}}
This file implements the interface defined in \code{cli.h} and 
furthermore defines some file-local functions to be used internally 
in the functions of the file.

\subsubsection*{\code{student.h}}
This file is probably the most important one. First and foremost it
defines the structure \code{student\_t} which is supposed to hold all 
data for one student. Secondly it defines \code{student\_list\_t}, which is a 
singly linked list structure. Such a structure contains a pointer to
a \code{student\_t} structure and a pointer to the next \code{student\_list\_t} element 
in the list. In this way a list with way more than 10000 students can be
constructed, but we only consume memory as needed.
The file defines 5 functions that can be used to manipulate such lists.
Last but not least, the file defines some functions that can be used
to encode and decode various parameters into/from the data field of \code{student\_t}.

\subsubsection*{\code{student.c}}
This files implements the interface defined in \code{student.h} and defines some
file-local functions necessary for the implementation.

\subsection*{Test}
\subsubsection*{Building the application}
\begin{verbatim}
sputnik% make clean
rm -f *.o
rm -f cudb
sputnik% make
gcc -Wall -g -c cli.c
gcc -Wall -g -c student.c
gcc -Wall -g -c main.c
gcc -Wall -g -o cudb cli.o student.o main.o
sputnik% 
\end{verbatim}

\subsubsection*{Normal use}
\begin{verbatim}
sputnik% ./cudb 

Welcome to CUDB - The C University Data Base

0: Halt
1: List all students
2: Add a new student

Enter action:
2

Enter name (4 characters only)
Hans

Enter start year (2009-2040):
2010

Enter start semester (0=Autumn/1=Spring):
0

Enter GPA (0-255)
50

Enter action:
2

Enter name (4 characters only)
Iben

Enter start year (2009-2040):
2011

Enter start semester (0=Autumn/1=Spring):
1

Enter GPA (0-255)
0

Enter action:
1

s0000 Hans 2010 Autumn 50
s0001 Iben 2011 Spring 0

Average GPA: 25.00

Enter action:
0
sputnik%
\end{verbatim}

\subsubsection*{Giving wrong input}
In this example, we first press CTRL+D (sending EOF to program),
and secondly we press ENTER (sending newline to program):

\begin{verbatim}
0: Halt
1: List all students
2: Add a new student

Enter action:
No such action exist!

Enter action:

No such action exist!

Enter action:

\end{verbatim}

Next we show some other types of wrong input.
Notice that we give a name with more than 4 characters - it
gets truncated:

\begin{verbatim}
Enter action:
2

Enter name (4 characters only)
Troels

Enter start year (2009-2040):
4
Not in allowed range!

Enter start year (2009-2040):
2010

Enter start semester (0=Autumn/1=Spring):
123
Not an option!

Enter start semester (0=Autumn/1=Spring):
asdf
Not an option!

Enter start semester (0=Autumn/1=Spring):
1

Enter GPA (0-255)
 ;asdf sld 23
Not in allowed range!

Enter GPA (0-255)

Not in allowed range!

Enter GPA (0-255)
40

Enter action:
1

s0000 Troe 2010 Spring 40

Average GPA: 40.00

Enter action:

\end{verbatim}
