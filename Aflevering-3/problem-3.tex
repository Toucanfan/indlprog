\section*{Problem 3}

The task was to implement the vector rally game -- which basically is
interactive, gamified vector addition.

A key feature of the game is that the walls of the map must not be crossed as
it would be equivalent of crashing the car. A reliable and general method of
detecting intersections between the cars movement vector and the walls is
needed.

The cars movement can be represented as line segment with following parameter
$$ (x, y) = (x_0, y_0) + t (\Delta x, \Delta y) $$
and in a similar fashion a wall segment can be represented as two points with
one as the origin and one as the direction vector
$$ (x, y) = (a_0, b_0) + s (\Delta a, \Delta b) $$
When these expressions are equal, the lines intersect. If the $t$ and $s$
factor are both satisfy $t, s \in [0,1]$ then the lines intersect between
their defining points -- in the game this translates to a collision.

This matrix represents the relationship and the equation

$$
\begin{bmatrix*}[c]
 \Delta x & -\Delta a & a_0 - x_0 & 0 \\
 \Delta y & -\Delta b & b_0 - y_0 & 0\\
\end{bmatrix*}
$$

Solving yields the following equations for calculating the 
intersecting point:

$$ t = \frac{-b_0 \Delta a + y_0 \Delta a + a_0 \Delta b - x_0 \Delta b}{\Delta b \Delta x - \Delta a \Delta y} $$
$$ s = \frac{-b_0 \Delta x + y_0 \Delta x + a_0 \Delta y - x_0 \Delta y}{\Delta b \Delta x - \Delta a \Delta y} $$

These calculations can be used and implemented directlt in Java since they
only contain simple calculations.

There is however a special case: If the two lines are parrallel there is no
need to test for intersections.

The procedure to detect an intersection is to:
\begin{enumerate}
\item Determine whether the lines segments are parallel
\item Caculate their deltas ($\Delta a$, $\Delta b$, $\Delta x$ and $\Delta y$)
\item Calculate $t$ and $s$ using above equations
\item If $t$ and $s$ both satisfy $t, s \in [0,1]$: Then the lines intersect
\end{enumerate}

This method is completely general and applies to any kind of line intersection
-- boundaries, goal lines or items. In the program it is implemented as the
method \code{intersects}.

\subsection*{Boundaries and goal lines}

The general line intersection detection makes it easy to implement boundaries
and goal lines.

When an obstacle is added to the map, the line segments corresponding to the
edges are added to an array. This global array containes each line segment as
a two-cell array containing start and end point. By looping over each element
in the boundary array, intersections can be detected. Since arrays are used
this puts an upper limit on the number of edges to account for (1024 in the
solution.)

The goal line uses the same code to check for intersections but here the
program must also detect if the car is driving in the right direction.
