\documentclass{scrartcl}

\usepackage[danish]{babel}
\usepackage[utf8]{inputenc}
\usepackage{mathtools}
\usepackage{fancyvrb}
%\usepackage{minted}

\fvset{xleftmargin=2em} % Indryk all fancyvrb kode lister

\newcommand\code[1]{\texttt{#1}}

\begin{document}

%\newminted{java}{linenos, obeytabs, tabsize=2}

\title{Afleveringsopgave 1}
\subtitle{01202 Indledende Programmering}
\author{
  Daniel Safari\\
  \texttt{s134110}
  \and
  Troels Mæhl Folke\\
  \texttt{s134061}
   \and
  Henrik Enggaard Hansen\\
  \texttt{s134099}
}
\date{16/2/2014}
\maketitle
\thispagestyle{empty}
\setcounter{page}{0}
\null
\vfill
\section*{Arbejdsfordeling}
Hvert medlem havde hovedansvaret for en opgave, dog spillede alle medlemer en aktiv rolle i hver opgave; kommentarer og forslag mm. Fordelingen var følgende:
\begin{itemize}
\setlength{\itemindent}{3em}
\item[Opgave 1:] Henrik Enggaard Hansen
\item[Opgave 2:] Troels Mæhl Folke
\item[Opgave 3:] Daniel Safari
\item[Rapport:]  Alle
\end{itemize}
\newpage
\section*{Opgave 1}
Da vores ønskede output indeholder to identiske linjer, så kan vi oprette en metode, som vi i stedet kan kalde, når den relevante linje ønskedes printet. Skulle vi eventuelt printe flere af disse linjer, vil dette gøre programmet mere overskueligt. For at udskrive \code{"} og \code{\textbackslash} bruger vi \emph{escape codes} for disse tegn.

\section*{Opgave 2}
Der skal defineres en variabel $x$ der får \code{println}-statement'et i \code{Opgave2.java}'s
\code{main()} metode til at udskrive "42"\ til konsollen.
$x$ kan ikke være en \emph{integer}, da resultatet af integer-divisionen i slutningen af
udtrykket i \code{println}-statement'et, $5/4*2=2$ sammenlagt med $1+3=4$ i
begyndelsen af udtrykket, ikke er
deleligt med 5 (som er antal gange $x$ lægges sammen i udtrykket).

Et alternativ er at lade $x$ være en \emph{double} med værdien 7.2. Dette vil
dog resultere i, at "42.0" skrives ud til konsollen og ikke blot "42".

For at opnå det ønskede resultat, skal man tænke i lidt andre baner: Udtrykket
på højresiden af $x$'erne giver som bekendt heltallet (\emph{integer}'en) 4 og
på venstresiden heltallet 2. Plus operatoren lægger gerne heltal sammen med
strenge (\emph{String}'s). Dette foregår ved først at konvertere heltallet til
en streng, og derefter concatenere de to strenge. Plus-operatoren evaluerer fra
venstre mod højre, efter eventuelle regneoperationer af højere prioritet er
udført (dvs. integer-divisionen i vores tilfælde). Hvis $x$ således defineres
til at være en tom streng, vil tallet 4 konverteres til en streng, og
concateneres med 5 tomme strenge, resulterende i strengen \code{"4"}. Denne vil da
blive concateneret med strengen \code{"2"} (resultatet af integer-divisionen,
koverteret til en streng), således at \code{"42"} skrives ud til konsollen.

\section*{Opgave 3}
Vi har et loop der genererer sekvensen \code{+-+-+-}, problemet er at den nuværende kode blot looper denne sekvens tre gange, og først derefter tilføjer \code{"+"} og skifter linje. Outputtet ser således ud:
\begin{Verbatim}
+-+-+-+-+-+-+-+-+-+
\end{Verbatim}
Vælger vi i stedet at inkludere pluset og linjeskiftet i vores ydre loop, så vil det ydre loop tre gange generere sekvensen \code{+-+-+-}, hvorefter der vil blive tilføjet et \code{+} og et efterfølgende linjeskift.
Vores output vil da se således ud:
\begin{Verbatim}
+-+-+-+
+-+-+-+
+-+-+-+
\end{Verbatim}
Hvilket jo netop var det ønskede output.

\end{document}
