\documentclass{scrartcl}

\usepackage[danish]{babel}
\usepackage[utf8]{inputenc}
\usepackage{mathtools}
\usepackage{fancyvrb}
%\usepackage{minted}

\fvset{xleftmargin=2em} % Indryk all fancyvrb kode lister

\newcommand\code[1]{\texttt{#1}}

\begin{document}

%\newminted{java}{linenos, obeytabs, tabsize=2}

\title{Afleveringsopgave 1}
\subtitle{01202 Indledende Programmering}
\author{
  Daniel Safari\\
  \texttt{s134110}
  \and
  Troels Mæhl Folke\\
  \texttt{s134061}
   \and
  Henrik Enggaard Hansen\\
  \texttt{s134099}
}
\date{16/2/2014}
\maketitle
\thispagestyle{empty}
\setcounter{page}{0}
\null
\vfill
\section*{Arbejdsfordeling}
Hvert medlem havde hovedansvaret for en opgave, dog spillede alle medlemer en aktiv rolle i hver opgave; kommentarer og forslag mm. Fordelingen var følgende:
\begin{itemize}
\setlength{\itemindent}{3em}
\item[Opgave 1:] Henrik Enggaard Hansen
\item[Opgave 2:] Daniel Safari
\item[Opgave 3:] Troels Mæhl Folke
\item[Rapport:]  Alle
\end{itemize}
\newpage
\section*{Opgave 1}
Da vores ønskede output indeholder to identiske linjer, så kan vi oprette en metode, som vi i stedet kan kalde, når den relevante linje ønskedes printet. Skulle vi eventuelt printe flere af disse linjer, vil dette gøre programmet mere overskueligt. For at udskrive \code{"} og \code{\textbackslash} bruger vi \emph{escape codes} for disse tegn.

\section*{Opgave 2}
Hvis ønsker at finde en konstant, x, der får følgende ligning til at gå op: 
\begin{align*}
1+3+x+x+x+x+x+5/4 \cdot 2 & = 42 \\
         4+5x+5/4 \cdot 2 & = 42
\end{align*}
Umiddelbart skulle man tro at løsningen var 7.1, men dette er ikke tilfældet. Årsagen er at compileren følger almen operatorprioritet. Resultatet af \code{5/4*2} bliver altså regnet først. \code{5} og \code{4} er to heltal (typen \emph{integer}), hvorved resultatet af den første division, \code{5/4}, bliver afkortet til 1. Så vores regnestykke ser faktisk således ud:
$$ 4+5x+2 = 42 $$
Hvor vi kan se løsningen til at være $x=7.2$.


\section*{Opgave 3}
Vi har et loop der genererer sekvensen \code{+-+-+-}, problemet er at den nuværende kode blot looper denne sekvens tre gange, og først derefter tilføjer \code{"+"} og skifter linje. Outputtet ser således ud:
\begin{Verbatim}
+-+-+-+-+-+-+-+-+-+
\end{Verbatim}
Vælger vi i stedet at inkludere pluset og linjeskiftet i vores ydre loop, så vil det ydre loop tre gange generere sekvensen \code{+-+-+-}, hvorefter der vil blive tilføjet et \code{+} og et efterfølgende linjeskift.
Vores output vil da se således ud:
\begin{Verbatim}
+-+-+-+
+-+-+-+
+-+-+-+
\end{Verbatim}
Hvilket jo netop var det ønskede output.

\end{document}
