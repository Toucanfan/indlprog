\documentclass{scrartcl}

\usepackage[danish]{babel}
\usepackage{fontspec}
\usepackage{mathtools}
\usepackage{fancyvrb}
%\usepackage{minted}

\fvset{xleftmargin=2em} % Indryk all fancyvrb kode lister

\begin{document}

%\newminted{java}{linenos, obeytabs, tabsize=2}

\title{Afleveringsopgave 1}
\subtitle{01202 Indledende Programmering}
\author{
  Daniel Safari\\
  \texttt{s134110}
  \and
  Troels Mæhl Folke\\
  \texttt{s134061}
   \and
  Henrik Enggaard Hansen\\
  \texttt{s134099}
}
\date{16/2/2014}
\maketitle

\section*{Opgave 1}
Da vores ønskede output indeholder to identiske linjer, så kan vi oprette en metode, som vi i stedet kan kalde, når den relevante linje ønskedes printet. Skulle vi eventuelt printe flere af disse linjer, vil dette gøre programmet mere overskueligt.

\section*{Opgave 2}
Hvis ønsker at finde en konstant, x, der får følgende ligning til at gå op: 
\begin{align*}
1+3+x+x+x+x+x+5/4*2 & = 42 \\
         4+5x+5/4*2 & = 42
\end{align*}
Umiddelbart skulle man tro at løsningen var 7.1, men dette er ikke tilfældet. Årsagen er at compileren følger almen operatorprioritet. Resultatet af $5/4*2$ bliver altså regnet først. 5,4 og 2 integers, hvorved resultatet af den første division, $5/4$, bliver afkortet til 2. Så vores regnestykke ser faktisk således ud:
$$ 4+5x+2 = 42 $$
Hvor vi kan se løsning til at være $x=7.2$.


\section*{Opgave 3}
Vi har et loop der genererer sekvensen \texttt{+-+-+-}, problemet er at den nuværende kode blot looper denne sekvens tre gange, og først derefter tilføjer \texttt{"+"} og skifter linje. Outputtet ser således ud:
\begin{Verbatim}
+-+-+-+-+-+-+-+-+-+
\end{Verbatim}
Vælger vi i stedet at inkludere pluset og linjeskiftet i vores ydre loop, så vil det ydre loop tre gange generere sekvensen \texttt{+-+-+-}, hvorefter der vil blive tilføjet et \texttt{+} og et efterfølgende linjeskift.
Vores output vil da se således ud:
\begin{Verbatim}
+-+-+-+
+-+-+-+
+-+-+-+
\end{Verbatim}
Hvilket jo netop var det ønskede output.

\end{document}