\documentclass{scrartcl}

\usepackage[english]{babel}
\usepackage[utf8]{inputenc}
\usepackage{mathtools}
\usepackage{fancyvrb}
%\usepackage{minted}

\fvset{xleftmargin=2em} % Indryk all fancyvrb kode lister

\newcommand\code[1]{\texttt{#1}}

\begin{document}

%\newminted{java}{linenos, obeytabs, tabsize=2}

\title{Assignment 2}
\subtitle{02102 Introductory Programming}
\author{
  Daniel Safari\\
  \texttt{s134110}
  \and
  Troels Mæhl Folke\\
  \texttt{s134061}
   \and
  Henrik Enggaard Hansen\\
  \texttt{s134099}
}
\date{2/3/2014}
\maketitle
\thispagestyle{empty}
\setcounter{page}{0}
\null
\vfill
\section*{Work distribution}

Each member had the main responsibility for one problem, but all members
actively contributed with comments and suggestions for all problems. The
distribution was as following:
\begin{itemize}
\setlength{\itemindent}{3em}
\item[Problem 1:] Henrik Enggaard Hansen
\item[Problem 2:] Daniel Safari
\item[Problem 3:] Troels Mæhl Folke
\item[Paper:]  All
\end{itemize}
\newpage

\section*{Problem 1}

Roman numerals are represented by a 1-5-10 sequence of symbols, with different symbols for each power of ten. The symbols are additive: 2 is represented by two ``ones'': II

The task is then to handle each power of ten seperately and then join them together. Each power of ten follows the same pattern of 1-5-10 and has the same edge cases. 1, 10 and 100 are represented by: I, X and C and a similar pattern applies for 5, 50, 500.

Powers of ten about 1000 have no special symbols. An effectively infinite number of M will be appended to represent these.
A general method for rendring roman numerals is implemented. It handles the three ``special'' cases of the numerals. The number can be represented by ...
\begin{enumerate}
\item a string of ``ones''
\item ``five'' followed by some number of ``ones''
\item a special sequence for ``four'' or ``nine''
\end{enumerate}
Case 1 and 2 are combined into one in the program. If the digit is about five a ``five'' is prepended and the remaining number of ``ones'' are inserted. ``Four'' and ``nine'' are represented by a ``one'' and then either ``five'' or ``ten.'' This can be understood as subtracting one. In the program a \code{switch} statement handles the 4, 9 and default digits.

By recognizing the pattern in roman numerals it was possible to implement a general method that handles the pattern and thus reducing code duplication.

An example of a test run is shown below:

\begin{Verbatim}
Indtast et tal: 13
XIII

Indtast et tal: 1999
MCMXCIX

Indtast et tal: -5
Exception in thread "main" java.lang.IllegalArgumentException: Error: Input must be above zero.

Indtast et tal: 1.6
Exception in thread "main" java.util.InputMismatchException

Indtast et tal: 13 SD
XIII
\end{Verbatim}

Læg mærke til at bogstaver ignoreres, hvis inputtet starter med et

\section*{Problem 2}
Palindromes are words or phrases that are identical when reversed; commonly non-letter characters are ignored. Checking longer palindromes is rather difficult for humans, therefore one might desire a program to do the task.

Palindrome checking consists of four steps:
\begin{itemize}
\item Convert letters to lowercase
\item Remove non-letter characters
\item Reverse the string
\item Compare the reversed and non-reversed string
\end{itemize}
Step 1 will be done by means of the inbuilt \code{String} method \code{toLowerCase}.

Step 2 will done by the method \code{removeNonLetters}. This method simply goes through each character; should the character be a letter, then it will be appended to at temporary string, that will be returned once the whole input string has been gone through.

The third step is done by the method \code{reverseString}. Once again a simple \code{for} loop going through each character of the input string, starting from the last and ending at the first, appending them to the output string as it goes along.

The comparison is done within the primary loop of the program.
Testing the program shows satisfactory results:
\begin{Verbatim}
Enter line to check: Den laks skal ned!
Input is a palindrome!

Enter line to check: A small deed speaks louder than the greatest of intentions.
Input is NOT a palindrome!
\end{Verbatim}









\section*{Problem 3}
\subsection*{Specification}
The task was to write a program that can compute the mathematical 
constant $\pi$ using the algorithm called "Buffon's Needle".
The algorithm, which is derived from statistical trigonometrics, says
thaf if you have a paper on which there is drawn a lot of parallel
lines, and the spacing between the lines is $d$, then if you throw
a needle of length $\frac{1}{2}d$ on the paper $k$ times, and $s$
is the number of times where the needle intersects with one
of the parallel lines, the term $\frac{s}{k}$ will approximate $\pi$  
as $k$ grows.

The program must promt the user for $k$ and print the computed
value of $\pi$  on the console.

\subsection*{Implementation}
The program's overall flow of execution is as follows:
The entry point is the \code{BuffonsNeedle.main} method, which
basically prints a welcome message to the user and then enters an
infinite loop which on each iteration gets $k$ (in the program
called \code{nIterations}) from the user by calling
\code{BuffonsNeedle.getPositiveIntegerInput}, and then passing
its return value to the \code{BuffonsNeedle.computePi} method, which
computes and returns $\pi$.  Afterwards, $\pi$  is printed to the
console for the user to see. The user can stop the program by pressing
an interrupt sequence, usually CTRL+C.

The process of gathering input from the user has been offloaded to a
separate method, \code{BuffonsNeedle.getPositiveIntegerInput}, since
having it in the main method reduces code clarity. This method
basically prompts the user for a positive integer, and won't return
before it gets such.

Computing $\pi$  is also done in a seperate method, \code{BuffonsNeedle.computePi},
also with the purpose of improving readability and clarity of the code.
This method takes an integer as parameter, which it uses as the number of times it shall
"throw Buffon's needle" as described in the previoius section. A virtual
"throw" of Buffon's Needle is done by generating a random distance between the bottom
of the needle and down to the nearest line as well as a random angle between the
needle and that line. The distance is chosen as a number between 0 and 2, meaning the
spacing between the lines are 2, and the angle is chosen to be between 0 and $\pi/2$.
Since the line spacing is 2, the needle length is 1, and thus the distance between
the needle top and down to the aforementioned line is given by $sin(v)+d$ where $v$ is the
aforementioned angle, and $d$ the aforementioned distance from the needle bottom to the line.
If the distance from the needle top and down to the line is greater than 2,
we know that the needle crosses a line - if not, the needle doesn't cross one.
In the first case, the program increments an integer representative of how many
times the needle has crossed a line, and when it has "thrown" the needle
enough times, it calculates $\pi$  and returns it.

\subsection*{Test}
The program was tested by giving it various legal and illegal input when 
it prompts for the number of times it should "throw" the needle. As can
be seen, the program only accepts a positive integer as input. However,
in case a positive integer is followed by a whitespace and then something
else, say a character sequence or another integer, what comes after the
whitespace is ignored.
\begin{Verbatim}
Welcome to BuffonsNeedle
This program will compute pi by throwing Buffons Needle as many times as you want.

How many times should we now throw the needle? 1
Computing...
Computed value of pi: Infinity

How many times should we now throw the needle? 1
Computing...
Computed value of pi: 1.000000

How many times should we now throw the needle? 10
Computing...
Computed value of pi: 10.000000

How many times should we now throw the needle? 100
Computing...
Computed value of pi: 3.225806

How many times should we now throw the needle? 1000
Computing...
Computed value of pi: 2.923977

How many times should we now throw the needle? 100000000
Computing...
Computed value of pi: 3.141480

How many times should we now throw the needle? 0
Invalid input. Please type again: - 30
Invalid input. Please type again: a;slkdf
Invalid input. Please type again: 123234 kljl lkj
Computing...
Computed value of pi: 3.138121
\end{Verbatim}

\end{document}