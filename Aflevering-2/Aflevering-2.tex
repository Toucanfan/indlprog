\documentclass{scrartcl}

\usepackage[english]{babel}
\usepackage[utf8]{inputenc}
\usepackage{mathtools}
\usepackage{fancyvrb}
%\usepackage{minted}

\fvset{xleftmargin=2em} % Indryk all fancyvrb kode lister

\newcommand\code[1]{\texttt{#1}}

\begin{document}

%\newminted{java}{linenos, obeytabs, tabsize=2}

\title{Assignment 2}
\subtitle{01202 Introductory Programming}
\author{
  Daniel Safari\\
  \texttt{s134110}
  \and
  Troels Mæhl Folke\\
  \texttt{s134061}
   \and
  Henrik Enggaard Hansen\\
  \texttt{s134099}
}
\date{2/3/2014}
\maketitle
\thispagestyle{empty}
\setcounter{page}{0}
\null
\vfill
\section*{Work distribution}

Each member had the main responsibility for one problem, but all members
actively contributed with comments and suggestions for all problems. The
distribution was as following:
\begin{itemize}
\setlength{\itemindent}{3em}
\item[Problem 1:] Henrik Enggaard Hansen
\item[Problem 2:] Troels Mæhl Folke
\item[Problem 3:] Daniel Safari
\item[Paper:]  Alle
\end{itemize}
\newpage
\section*{Problem 1}

To write the number $n$ as a roman numeral it is necessary to consider if the
number falls in one of three cases:
\begin{enumerate}
\item The number consists of a string of ones
\item The number is five followed by some number of ones
\item The number is either four or nine
\end{enumerate}

The simplest case is a string of ones: Simply print a string consisting of $n$
one-symbols.

For case two we need to print a symbol representing five followed by the
remaining amount of symbols for ones

The first is a special case, where one is prepended to either five or ten,
representing a subtraction of one.

In the provided program, case one and two are combined into one. If the number
is equal to or more than five, a five-symbol is prepended. The number is then
decreased by five and the number of ones are added. This code takes care of
both cases.

Because of the regularity of roman numerals, this pattern can be reused across
all powers of ten (up to 1000.) In the code this is implemented in the
\codr{generalNumeral} method.

\section*{Problem 2}



\section*{Problem 3}



\end{document}
 
