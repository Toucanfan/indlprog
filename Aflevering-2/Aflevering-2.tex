\documentclass{scrartcl}

\usepackage[english]{babel}
\usepackage[utf8]{inputenc}
\usepackage{mathtools}
\usepackage{fancyvrb}
%\usepackage{minted}

\fvset{xleftmargin=2em} % Indryk all fancyvrb kode lister

\newcommand\code[1]{\texttt{#1}}

\begin{document}

%\newminted{java}{linenos, obeytabs, tabsize=2}

\title{Assignment 2}
\subtitle{01202 Introductory Programming}
\author{
  Daniel Safari\\
  \texttt{s134110}
  \and
  Troels Mæhl Folke\\
  \texttt{s134061}
   \and
  Henrik Enggaard Hansen\\
  \texttt{s134099}
}
\date{2/3/2014}
\maketitle
\thispagestyle{empty}
\setcounter{page}{0}
\null
\vfill
\section*{Work distribution}

Each member had the main responsibility for one problem, but all members
actively contributed with comments and suggestions for all problems. The
distribution was as following:
\begin{itemize}
\setlength{\itemindent}{3em}
\item[Problem 1:] Henrik Enggaard Hansen
\item[Problem 2:] Troels Mæhl Folke
\item[Problem 3:] Daniel Safari
\item[Paper:]  Alle
\end{itemize}
\newpage
\section*{Problem 1}

Roman numerals are represented by a 1-5-10 sequence of symbols, with different symbols for each power of ten. The symbols are additive: 2 is represented by two ``ones'': II

The task is then to handle each power of ten seperately and then join them together. Each power of ten follows the same pattern of 1-5-10 and has the same edge cases. 1, 10 and 100 are represented by: I, X and C and a similar pattern applies for 5, 50, 500.

Powers of ten about 1000 have no special symbols. An effectively infinite number of M will be appended to represent these.

A general method for rendring roman numerals is implemented. It handles the three ``special'' cases of the numerals. The number can be represented by ...
\begin{enumerate}
\item a string of ``ones''
\item ``five'' followed by some number of ``ones''
\item a special sequence for ``four'' or ``nine''
\end{enumerate}

Case 1 and 2 are combined into one in the program. If the digit is about five a ``five'' is prepended and the remaining number of ``ones'' are inserted. ``Four'' and ``nine'' are represented by a ``one'' and then either ``five'' or ``ten.'' This can be understood as subtracting one. In the program a \code{switch} statement handles the 4, 9 and default digits.

By recognizing the pattern in roman numerals it was possible to implement a general method that handles the pattern and thus reducing code duplication.

\section*{Problem 2}



\section*{Problem 3}



\end{document}
 
