\documentclass{scrartcl}

\usepackage[english]{babel}
\usepackage[utf8]{luainputenc}
\usepackage{mathtools}
\usepackage{fancyvrb}
%\usepackage{minted}

\fvset{xleftmargin=2em} % Indryk all fancyvrb kode lister

\newcommand\code[1]{\texttt{#1}}

\begin{document}

%\newminted{java}{linenos, obeytabs, tabsize=2}

\title{Assignment 2}
\subtitle{01202 Introductory Programming}
\author{
  Daniel Safari\\
  \texttt{s134110}
  \and
  Troels Mæhl Folke\\
  \texttt{s134061}
   \and
  Henrik Enggaard Hansen\\
  \texttt{s134099}
}
\date{2/3/2014}
\maketitle
\thispagestyle{empty}
\setcounter{page}{0}
\null
\vfill
\section*{Work distribution}

Each member had the main responsibility for one problem, but all members
actively contributed with comments and suggestions for all problems. The
distribution was as following:
\begin{itemize}
\setlength{\itemindent}{3em}
\item[Problem 1:] Henrik Enggaard Hansen
\item[Problem 2:] Daniel Safari
\item[Problem 3:] Troels Mæhl Folke
\item[Paper:]  All
\end{itemize}
\newpage
\section*{Problem 1}



\section*{Problem 2}



\section*{Problem 3}
\subsection*{Specification}
The task was to write a program that can compute the mathematical 
constant $\pi$ using the algorithm called "Buffon's Needle".
The algorithm, which is derived from statistical trigonometrics, says
thaf if you have a paper on which there is drawn a lot of parallel
lines, and the spacing between the lines is $d$, then if you throw
a needle of length $\frac{1}{2}d$ on the paper $k$ times, and $s$
is the number of times where the needle intersects with one
of the parallel lines, the term $\frac{s}{k}$ will approximate $\pi$  
as $k$ grows.

The program must promt the user for $k$ and print the computed
value of $\pi$  on the console.

\subsection*{Implementation}
The program's overall flow of execution is as follows:
The entry point is the \code{BuffonsNeedle.main} method, which
basically prints a welcome message to the user and then enters an
infinite loop which on each iteration gets $k$ (in the program
called \code{nIterations}) from the user by calling
\code{BuffonsNeedle.getPositiveIntegerInput}, and then passing
its return value to the \code{BuffonsNeedle.computePi} method, which
computes and returns $\pi$.  Afterwards, $\pi$  is printed to the
console for the user to see. The user can stop the program by pressing
an interrupt sequence, usually CTRL+C.

The process of gathering input from the user has been offloaded to a
separate method, \code{BuffonsNeedle.getPositiveIntegerInput}, since
having it in the main method reduces code clarity. This method
basically prompts the user for a positive integer, and won't return
before it gets such.

Computing $\pi$  is also done in a seperate method, \code{BuffonsNeedle.computePi},
also with the purpose of improving readability and clarity of the code.
This method takes an integer as parameter, which it uses as the number of times it shall
"throw Buffon's needle" as described in the previoius section. A virtual
"throw" of Buffon's Needle is done by generating a random distance between the bottom
of the needle and down to the nearest line as well as a random angle between the
needle and that line. The distance is chosen as a number between 0 and 2, meaning the
spacing between the lines are 2, and the angle is chosen to be between 0 and $\pi/2$.
Since the line spacing is 2, the needle length is 1, and thus the distance between
the needle top and down to the aforementioned line is given by $sin(v)+d$ where $v$ is the
aforementioned angle, and $d$ the aforementioned distance from the needle bottom to the line.
If the distance from the needle top and down to the line is greater than 2,
we know that the needle crosses a line - if not, the needle doesn't cross one.
In the first case, the program increments an integer representative of how many
times the needle has crossed a line, and when it has "thrown" the needle
enough times, it calculates $\pi$  and returns it.

\subsection*{Test}
The program was tested by giving it various legal and illegal input when 
it prompts for the number of times it should "throw" the needle. As can
be seen, the program only accepts a positive integer as input. However,
in case a positive integer is followed by a whitespace and then something
else, say a character sequence or another integer, what comes after the
whitespace is ignored.
\begin{Verbatim}
Welcome to BuffonsNeedle

This program will compute pi by throwing Buffons Needle as many times as you want.
How many times should we throw the needle? 1
Computing...
Computed value of pi: Infinity

How many times should we now throw the needle? 1
Computing...
Computed value of pi: Infinity

How many times should we now throw the needle? 1
Computing...
Computed value of pi: 1.000000

How many times should we now throw the needle? 10
Computing...
Computed value of pi: 10.000000

How many times should we now throw the needle? 100
Computing...
Computed value of pi: 3.225806

How many times should we now throw the needle? 1000
Computing...
Computed value of pi: 2.923977

How many times should we now throw the needle? 10000
Computing...
Computed value of pi: 3.099814

How many times should we now throw the needle? 100000
Computing...
Computed value of pi: 3.159059

How many times should we now throw the needle? 1000000
Computing...
Computed value of pi: 3.147198

How many times should we now throw the needle? 10000000
Computing...
Computed value of pi: 3.141926

How many times should we now throw the needle? 100000000
Computing...
Computed value of pi: 3.141480

How many times should we now throw the needle? 0
Invalid input. Please type again: -1
Invalid input. Please type again: -20
Invalid input. Please type again: - 30
Invalid input. Please type again: a;slkdf
Invalid input. Please type again: aks asda; laskdf asd; asdlk kdfjkd
Invalid input. Please type again: aslkdf 234234 asdlkfj
Invalid input. Please type again: 123234 kljl lkj
Computing...
Computed value of pi: 3.138121

How many times should we now throw the needle? ^[[A^[[A
Invalid input. Please type again:  
 
d
Invalid input. Please type again: 




4
Computing...
Computed value of pi: Infinity

How many times should we now throw the needle? 




6
Computing...
Computed value of pi: 6.000000

How many times should we now throw the needle? ^C
\end{Verbatim}

\end{document}
 
